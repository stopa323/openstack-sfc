\chapter{OpenStack}
\label{cha:openstack}

OpenStack is a numerous set of open-source software tools that use pooled virtual resources to build, manage and orchestrate public and private clouds. Modular architecture as a core design tenet facilitates scalability and elasticity and enables operator to design and deploy unique platform based on required type of services or virtualized resources. Six of those modules handle the core cloud-computing functionality of compute, storage, networking, identity and image services. \par
In this chapter overall architecture along with core and several optional projects, essential for deploying environment capable of service chaining, is discussed. Section \ref{sec:openstack_arch} introduces general view of OpenStack ecosystem and its core components. Section \ref{sec:telemetry} describes telemetry project componenents (Ceilometer, Aodh and Gnocchi) used for monitoring VNFs and alarming. Chapter ends with compendious characterisation of Heat and Mistral projects employed for orchestration.

%---

\section{Architecture}
\label{sec:openstack_arch}

\cite{MAN02}

%---

\section{Telemetry}
\label{sec:telemetry}

\cite{ART03}

%---

\section{Orchestration}
\label{sec:orchestration}
